%!TEX root = /Users/gilesb/programming/presentations/yycruby-2015-05/metaprogramming-presentation.tex
\usepackage{etex}
\usepackage[english]{babel}
% or whatever

\usepackage[all]{xy}
\usepackage[latin1]{inputenc}
% or whatever

\usepackage{palatino,courier}
\usepackage{amsfonts,amssymb}
\usepackage[mathscr]{euscript}
\usepackage{stmaryrd}
\usepackage{eulervm}

\usepackage{proof}
\usepackage{amsmath}
\usepackage{xspace}

\usepackage{graphicx}
% \usepackage{tikz}
\usepackage{listings}
\usepackage{color}
\usepackage{hyperref}

\lstloadlanguages{Ruby}
\lstset{language=Ruby}
\lstset{% general command to set parameter(s)
basicstyle=\footnotesize\ttfamily , % print whole listing small
keywordstyle=\color{black}\bfseries,
% underlined bold black keywords
identifierstyle=\color{blue}, % nothing happens
commentstyle=\color{gray}, % white comments
stringstyle=\ttfamily, % typewriter type for strings
showstringspaces=false} %
% \usetikzlibrary{calc,arrows,positioning,shapes.geometric,fit,decorations.markings}
%
% \pgfdeclarelayer{edgelayer}
% \pgfdeclarelayer{nodelayer}
% \pgfsetlayers{edgelayer,nodelayer,main}
%
% \tikzset{ node distance=.1mm, inner sep=0.5mm}
%
% \tikzstyle{delta}=[isosceles triangle, isosceles triangle apex angle=70, draw, shape border rotate=90, minimum size=2mm]
% \tikzstyle{eta}=[circle, draw, minimum size=2mm]
% \tikzstyle{epsilon}=[circle, draw, fill=black!75, minimum size=2mm]
% \tikzstyle{nabla}=[isosceles triangle, isosceles triangle apex angle=70, draw, shape border rotate=270, minimum size=2mm]
% \tikzstyle{map}=[rectangle,draw]
% \tikzstyle{nothing}=[rectangle,draw=white]




\newcommand{\bi}{\begin{itemize}}
\newcommand{\ei}{\end{itemize}}
\newcommand{\be}{\begin{enumerate}}
\newcommand{\ee}{\end{enumerate}}
\newcommand{\bd}{\begin{description}}
\newcommand{\ed}{\end{description}}
\newcommand{\itembf}[1]{\item{\textbf{#1}}}
\newcommand{\itemem}[1]{\item{\emph{#1}}}
\newcommand{\itemtt}[1]{\item{\texttt{#1}}}

\newcommand{\injsub}[2]{\genfrac{}{}{0pt}{3}{#1}{#2}}

\newcommand{\restr}[1]{\overline{#1}}
\newcommand{\rst}[1]{\restr{#1}}

\newcommand{\rg}[1]{\hat{#1}}
\newcommand{\wrg}[1]{\widehat{#1}}

\newcommand{\cp}[1]{\amalg_{#1}}
\newcommand{\cpa}{\cp{1}}
\newcommand{\cpb}{\cp{2}}

\newcommand{\icp}[1]{\inv{\cp{#1}}}
\newcommand{\icpa}{\icp{1}}
\newcommand{\icpb}{\icp{2}}

\newcommand{\scp}[1]{\cp{#1}^{*}}
\newcommand{\scpa}{\scp{1}}
\newcommand{\scpb}{\scp{2}}


\let\*\otimes
\let\+\oplus
\let\<\langle
\let\>\rangle

\newcommand{\meet}{\cap}
\newcommand{\axiom}[2]{\ensuremath{[\text{\bfseries {#1}.{#2}}]}\xspace}

\newcommand{\rstaxiom}[1]{\axiom{R}{#1}}

\newcommand{\rgaxiom}[1]{\axiom{RR}{#1}}
\newcommand{\rone}{\rstaxiom{1}}
\newcommand{\rtwo}{\rstaxiom{2}}
\newcommand{\rthree}{\rstaxiom{3}}
\newcommand{\rfour}{\rstaxiom{4}}

\newcommand{\rrone}{\rgaxiom{1}}
\newcommand{\rrtwo}{\rgaxiom{2}}
\newcommand{\rrthree}{\rgaxiom{3}}
\newcommand{\rrfour}{\rgaxiom{4}}


\newcommand{\perpvv}[2]{{}_{{}_{#1}}{\perp_{{}_{#2}}}}
\newcommand{\perpab}{\perpvv{A}{B}}
\newcommand{\perpba}{\perpvv{B}{A}}
\newcommand{\perpcb}{\perpvv{C}{B}}
\newcommand{\tperp}{\perpvv{}{\oplus}}

\newcommand{\djoin}{\sqcup}
\newcommand{\djoinbig}{\bigsqcup}

\DeclareMathOperator{\diag}{\scriptstyle{diag}}

\newcommand{\altjoin}{\,\square\,}
\newcommand{\disjoint}{\perp}
\newcommand{\invdisjoint}{\disjoint^0}
\newcommand{\invjoin}{\join^0}
\newcommand{\tjoin}{\djoin_{{_{\oplus}}}}
\newcommand{\intersection}{\cap}
\newcommand{\zeroob}{\ensuremath{\mathbf{0}}}


\DeclareMathOperator{\tjdown}{\triangledown}
\DeclareMathOperator{\tjup}{\vartriangle}
\DeclareMathOperator{\gtjdown}{\overline{\triangledown}}
\DeclareMathOperator{\gtjup}{\underline{\vartriangle}}


\newcommand{\ocdperp}{\,\underline{\perp}\,}
\newcommand{\eocdperp}{\,\underline{\underline{\perp}}\,}
\newcommand{\ocdperpsub}[1]{\,\underline{\perp}_{{}_{#1}}\,}
\newcommand{\cdperp}{\perp}


\newcommand{\com}[1]{\ensuremath{c_{#1}}}
\newcommand{\comp}{\com{\+}}
\newcommand{\assoc}[1]{\ensuremath{a_{#1}}}
\newcommand{\assocp}{\assoc{\+}}


\newcommand{\ea}{e_1}
\newcommand{\eb}{e_2}
\newcommand{\xa}{x_1}
\newcommand{\xb}{x_2}

\newcommand{\wtf}{\ensuremath{\widetilde{(\_)}}\xspace}
\newcommand{\wt}[1]{\ensuremath{\widetilde{#1}}\xspace}


\newcommand{\spl}[2]{\ensuremath{\text{K}_{#1}(#2)}}
\newcommand{\inv}[1]{\ensuremath{{#1}^{(-1)}}}

\newcommand{\uleft}[1]{\ensuremath{u_{#1}^l}}
\newcommand{\uright}[1]{\ensuremath{u_{#1}^r}}
\newcommand{\usl}{\uleft{\*}}
\newcommand{\usr}{\uright{\*}}
\newcommand{\upl}{\uleft{\+}}
\newcommand{\upr}{\uright{\+}}

\newcommand{\Xt}{\ensuremath{\widetilde{\X}}\xspace}

\newcommand{\xtdmn}[2]{\ensuremath{{#1}_{|{#2}}}}


\newcommand{\xequiv}[1]{\ensuremath{\overset{\scriptscriptstyle #1}{\simeq}}}
\newcommand{\quest}[1]{\par{\Large \textbf{ (Question):}#1\textbf{(end question.)}}}
\newcommand{\noteb}[1]{{\Large \textbf{(Internal Note):}#1\textbf{(end note.)}}}


\newcommand{\nm}{\ensuremath{n\times{}m}}
\newcommand{\mn}{\ensuremath{m\times{}n}}

\newcommand{\specialcat}[1]{\textsc{#1}\xspace}
\newcommand{\ltrcat}[1]{\ensuremath{\mathfrak{#1}}\xspace}
\newcommand{\ltrcatbb}[1]{\ensuremath{\mathbb{#1}}\xspace}
\newcommand{\B}{\ltrcatbb{B}}
\newcommand{\C}{\ltrcatbb{C}}
\newcommand{\D}{\ltrcatbb{D}}
\newcommand{\N}{\ltrcatbb{N}}
\newcommand{\X}{\ltrcatbb{X}}
\newcommand{\Y}{\ltrcatbb{Y}}
\newcommand{\Z}{\ltrcatbb{Z}}
\newcommand{\T}{\ltrcatbb{T}}

\newcommand{\dmap}[1]{\ensuremath{{#1}^{\Delta}_{\nabla}}\xspace}

\newcommand{\open}[1]{\ensuremath{\mathcal{O}({#1})}}
\newcommand{\undef}{\uparrow}


\newcommand{\Hil}{\ensuremath{\mathcal{H}}\xspace}
\newcommand{\topcat}{\specialcat{Top}}
\newcommand{\topcatp}{\ensuremath{\topcat_p}\xspace}

\newcommand{\sets}{\specialcat{Sets}}
\newcommand{\psets}{\specialcat{PSets}}
\newcommand{\Par}{\specialcat{Par}}
\newcommand{\pinj}{\specialcat{PInj}}
\newcommand{\rel}{\specialcat{Rel}}
\newcommand{\mon}{\specialcat{Mon}}
\newcommand{\ring}{\specialcat{Rng}}
\newcommand{\cring}{\specialcat{CRng}}
\newcommand{\cat}{\textbf{\specialcat{Cat}}}
\newcommand{\poset}{\specialcat{Poset}}
\newcommand{\preorder}{\specialcat{Preorder}}

\newcommand{\CFrob}{\ensuremath{\specialcat{CFrob}(\X)}\xspace}

\newcommand{\Mstab}{\ensuremath{\mathcal{M}}\xspace}

\newcommand{\obj}[1]{\ensuremath{#1_{obj}}}
\newcommand{\bottom}[1]{\perp_{#1}}
\newcommand{\finpower}{\mathscr{P}_{fin}}

\newcommand{\category}[4]{%
\begin{description}%
\item[\textbf{Objects: }]{#1}%
\item[\textbf{Maps: }]{#2}%
\item[\textbf{Identity: }]{#3}%
\item[\textbf{Comp.: }]{#4}%
\end{description}%
%
}
\newcommand{\rcategory}[5]{%
\begin{description}%
\item[\textbf{Objects: }]{#1}%
\item[\textbf{Maps: }]{#2}%
\item[\textbf{Identity: }]{#3}%
\item[\textbf{Comp.: }]{#4}%
\item[\textbf{Restrict: }]{#5}%
\end{description}%
%
}
\newcommand{\categoryom}[2]{%
\begin{description}%
\item[\textbf{Objects: }]{#1}%
\item[\textbf{Maps: }]{#2}%
\end{description}%
}

\newcommand{\pfcategory}[3]{%
\begin{description}%
\item{\textbf{Well-Defined: }}{#1}%
\item{\textbf{Identities: }}{#2}%
\item{\textbf{Associativity: }}{#3}%
\end{description}}


% general math symbols
\newcommand{\union}{\ensuremath{\bigcup}}
\newcommand{\disjunion}{\ensuremath{\sqcup}}
\newcommand{\intersect}{\ensuremath{\bigcap}}
\newcommand{\logor}{\ensuremath{\lor}}
\newcommand{\logand}{\ensuremath{\land}}
\newcommand{\lognand}{\ensuremath{\barwedge}}
\newcommand{\natmap}{\ensuremath{\Rightarrow}}
\newcommand{\fctrmap}{\ensuremath{\to}}
\newcommand{\fnctrmap}{\fctrmap}
\newcommand{\fmap}{\ensuremath{\to}}
\newcommand{\produces}{\ensuremath{\to}}
\newcommand{\ladjoint}{\ensuremath{\dashv}}
\newcommand{\pproj}{\ensuremath{\preceq_p}}

\newcommand{\tr}[1]{\ensuremath{\text{tr}(#1)}}

\newcommand{\colvec}[1]{\ensuremath{
\begin{pmatrix}#1\end{pmatrix}}}

\newcommand{\tbtmatrix}[4]{\ensuremath{
\begin{pmatrix}
#1&#2\\
#3&#4
\end{pmatrix}
}}

\newcommand{\quadmatrix}[4]{\ensuremath{
\left(
\begin{array}{c|c}
#1&#2\\
\hline
#3&#4
\end{array}
\right)
}}
%--------------------------------------------------------------------------------
% Macros for Turing categories

\newcommand{\name}[1]{\ensuremath{{}^\ulcorner\!\!#1^\urcorner}}
\newcommand{\iname}[1]{\name{#1}}   %{\ensuremath{{}_\llcorner#1_{\!\lrcorner}}}
\newcommand{\code}[1]{\ensuremath{#1_{\bullet}}}

\newcommand{\tur}[2]{\ensuremath{\tau_{{#1},{#2}}}}
\newcommand{\txy}{\tur{X}{Y}}
\newcommand{\multiapp}[2]{\ensuremath{{#1}^{(#2)}}}
\newcommand{\multibullet}[1]{\multiapp{\bullet}{#1}}
\newcommand{\imultiapp}[2]{\ensuremath{{#1}^{[#2]}}}
\newcommand{\imultibullet}[1]{\imultiapp{\bullet}{#1}}


\newcommand{\compa}{\ensuremath{\specialcat{Comp}(A)}\xspace}
\newcommand{\compn}{\ensuremath{\specialcat{Comp}(\N)}\xspace}
\DeclareMathOperator*{\definedas}{:\!=}
\newcommand{\Inv}[1]{\ensuremath{\mathbf{INV}({#1})}\xspace}
\newcommand{\retract}{\ensuremath{\triangleleft}}
\newcommand{\retractmaps}[2]{\ensuremath{\triangleleft_{#1}^{#2}}}


\title{Metaprogramming in Ruby}
\subtitle{Putting the FUN in conFUsionN}

\author[Dr. Brett~Giles]{\protect{Dr.~Brett~Giles}}

% - Use the \inst{?} command only if the authors have different
%   affiliation.

\institute[YYCRuby] % (optional, but mostly needed)
{  YYCRuby Meetup}
% - Use the \inst command only if there are several affiliations.
% - Keep it simple, no one is interested in your street address.

\date
{2015-05}

\subject{Metaprogramming}
% This is only inserted into the PDF information catalog. Can be left
% out.

% Delete this, if you do not want the table of contents to pop up at
% the beginning of each subsection:
%\AtBeginSubsection[]
% {
%  \begin{frame}<beamer>{Outline}
%    \tableofcontents[currentsection,currentsubsection]
%  \end{frame}
% }


% If you wish to uncover everything in a step-wise fashion, uncomment
% the following command:

%\beamerdefaultoverlayspecification{<+->}


\begin{document}

\begin{frame}
  \titlepage
\end{frame}

\section{Metaprogramming: to be or to possibly be?}
\begin{frame}\frametitle{Metaprogramming}
  The Art and Science of programs that write programs:
\begin{itemize}
  \item Compilers, Lex, YACC
  \item Macros in languages like C
  \item Templates in C++
  \item Interpreters that allow you to evaluate strings
\end{itemize}
\end{frame}
\begin{frame}[fragile]\frametitle{In Ruby...}
  \begin{itemize}
    \item \lstinline*define_method, define_singleton_method*
    \item \lstinline*method_missing*
    \item Reflection methods (\lstinline*methods, respond_to?, send, __send__, public_send*)
    \item The eval/exec methods (\lstinline*class_eval, instance_eval, eval*) and (\lstinline*class_exec, instance_exec, exec*)
    \item Hook methods (\lstinline*included, prepended, extended, inherited*)
    \item Reopening Classes (Monkeypatching!)
  \end{itemize}
\end{frame}
\begin{frame}\frametitle{Why or why not metaprogramming?}
  Uses and goodies:
  \begin{itemize}
    \item Domain Specific Languages
    \item DRY code
    \item Nice fit for the Adaptor pattern with external libraries
    \item Dynamic dispatch
  \end{itemize}
  Not so fun things:
  \begin{itemize}
    \item Code obfuscation
    \item Clever code - Too many levels of indirection
    \item Naming!
  \end{itemize}

\end{frame}
\begin{frame}\frametitle{Where do we start?}
  Common metaprogramming tasks:
  \begin{itemize}
    \item Create a class macro
    \item Handle a variety of similarly named methods doing the ``same'' thing
    \item Evaluate some code in the context of an object
  \end{itemize}
\end{frame}
\begin{frame}[fragile]\frametitle{The object hierarchy in Ruby}
  \begin{center}
  \includegraphics[scale=.6]{diagrams/object-hierarchy.png}
  \end{center}
\end{frame}
\begin{frame}[fragile]\frametitle{Create a class macro}
\begin{lstlisting}

Class Attr
  def self.my_attr(attribute)
    define_method(attribute) { instance_eval("@#{attribute}") }
    define_method("#{attribute}=") do |value|
      instance_eval("@#{attribute} = #{value}")
    end
  end
end

\end{lstlisting}
\end{frame}
\begin{frame}[fragile]\frametitle{Similar methods - part 1}
\begin{lstlisting}
class IamAnAdaptor
  def initialize
    @adaptee = SomeClass.new
    @adaptee.methods.select { |m| m =~ /work.*/}.each do |m|
      define_method "sc_#{m}" do
        puts 'Work it!'
        @adaptee.send(m)
      end
    end
  end
end
\end{lstlisting}
\end{frame}
\begin{frame}[fragile]\frametitle{Similar methods - part 2}
\begin{lstlisting}
class IamAnAdaptor
  def initialize
    @adaptee = SomeClass.new
  end

  def method_missing(m, *args)
    md = m.to_s.match(/sc_(work.*)/)
    if md && @adaptee.methods.include?(md[1].to_sym)
      puts 'Work it!'
      @adaptee.send(md[1])
    else
      super
    end
  end
end
\end{lstlisting}
\end{frame}
\begin{frame}[fragile]\frametitle{Similar methods - part 3}
\begin{lstlisting}
class IamAnAdaptor
  def initialize
    @adaptee = SomeClass.new
  end

  def method_missing(m, *args)
    md = m.to_s.match(/sc_(work.*)/)
    if md && @adaptee.methods.include?(md[1].to_sym)
      self.class.class_eval do
        define_method m do
          puts 'Work it!'
          @adaptee.send(md[1])
        end
      end
      self.send(m)
    else
      super
    end
  end
end
\end{lstlisting}
\end{frame}

\begin{frame}[fragile]\frametitle{Real world}
  Active Record 4.2.1, associations/builder/association.rb
\begin{lstlisting}
  def self.define_readers(mixin, name)
    mixin.class_eval <<-CODE, __FILE__, __LINE__ + 1
      def #{name}(*args)
        association(:#{name}).reader(*args)
      end
    CODE
  end

  def self.define_writers(mixin, name)
    mixin.class_eval <<-CODE, __FILE__, __LINE__ + 1
      def #{name}=(value)
        association(:#{name}).writer(value)
      end
    CODE
  end
\end{lstlisting}
\end{frame}
\begin{frame}[fragile]\frametitle{Real world}
  rspec core $3.2$, memoized\_helpers.rb
\begin{lstlisting}[basicstyle=\scriptsize\ttfamily]
def let(name, &block)
  # We have to pass the block directly to `define_method` to
  # allow it to use method constructs like `super` and `return`.
  raise "#let or #subject called without a block" if block.nil?
  MemoizedHelpers.module_for(self)
    .__send__(:define_method, name, &block)

  # Apply the memoization. The method has been defined in an ancestor
  # module so we can use `super` here to get the value.
  if block.arity == 1
    define_method(name) { __memoized.fetch(name) { |k|
      __memoized[k] = super(RSpec.current_example, &nil) } }
  else
    define_method(name) { __memoized.fetch(name) { |k|
      __memoized[k] = super(&nil) } }
  end
end
\end{lstlisting}
\end{frame}
\begin{frame}\frametitle{Recent blogs, more details}
  Books:
  \begin{itemize}
    \item \href{https://pragprog.com/book/ruby4/programming-ruby-1-9-2-0}{Programming Ruby, Chapter 24}
    \item \href{https://pragprog.com/book/ppmetr2/metaprogramming-ruby-2}{MetaProgramming Ruby 2}
  \end{itemize}
  Blogs / online:
  \begin{itemize}
    \item \href{http://ruby-metaprogramming.rubylearning.com/}{Ruby learning's metaprogramming}
    \item \href{http://www.sitepoint.com/rubys-important-hook-methods/}{Sitepoint: - Hook methods (included, ...)}
    \item \href{https://www.codeschool.com/blog/2015/04/24/7-deadly-sins-of-ruby-metaprogramming/}{CodeSchool - 7 deadly sins of metaprogramming}
  \end{itemize}
\end{frame}
\begin{frame}\frametitle{Seven sins?}
  \begin{itemize}
    \item Using \lstinline|method_missing| as your first option
    \item Not overriding \lstinline|respond_to_missing?|
    \item Not handling all cases!
    \item Using \lstinline|define_method| when it is not needed(Hmmm...)
    \item Changing the semantics when opening classes. (e.g., redefining \lstinline|:+| to add 5 to the result)
    \item Depending on who is using you (Depend down, not up)
    \item Deep nesting (e.g., RSpec tests)
  \end{itemize}
\end{frame}
\begin{frame}\frametitle{Code exercise}
  \begin{center}
    \url{https://github.com/drogar/meta-yycruby-code}
  \end{center}
\end{frame}

\end{document}



%%% Local Variables:
%%% mode: latex
%%% TeX-master: "withnotes"
%%% End:
